\documentclass[11pt]{article}

\begin{document}

\title{Some Notes on 3D Raytracing}
\author{Peter Erwin}

\maketitle

\section{Basics of Light Interaction with Surfaces}

We can divide the interactions of light into two basic types:
\textbf{radiative transfer through a medium} and \textbf{scattering}.
Scattering can take place in two ways: interactions with small particles
in a medium (e.g., interactions of light with fog, etc.), and scattering
at a boundary between media with different indices of refractions.


\subsection{Scattering (Reflection and Refraction) at a Planar Boundary}

Scattering at a (locally planar) boundary is \textit{most} of the
interesting effects in 3D rendering: it involves light reflecting,
scattering, and/or being refracted when light hits an object. The basic
math for this was worked out by Fresnel in the early 19th Century, and
is codified in the Fresnel equations. In simple terms (ignoring the role
of polarization), a light ray that intersects a planar boundary is
scattered in two discrete ways: reflection away from the boundary and
refraction through the boundary. Energy conservation means that the
sum of reflection and refraction must = incoming light.

\textit{Direct (or specular) reflection}, on a locally planar level, follows the
standard angle of incidence = angle of reflection model. For perfectly
flat surfaces, this is a single angle. For surfaces with microstructure,
the surfaces can be treated as a collection of locally flat regions (the
``microfacets'' model); the combined direct reflection will involve a
small range of angles about the mean direct reflection: generalzied
\textit{specular reflection.}

\textit{Refraction:} Refracted light travels into the second object/medium (at
an angle determined by Snell's Law). It then undergoes scattering and absorption
within the second object/medium.

For metals (conductors), the refracted light is absorbed almost immediately (apparently the free
electrons ensure this), and can thus be ignored (except, of course, in terms of
energy conservation: the refracted light is lost).

For dielectrics, the refracted light travels through the object/medium
and is subject to absorption and scattering. \textit{Some} of the
scattered light makes it way back out of the surface (e.g., after
multiple scatterings). This is knowns as ``subsurface scattering'', and
is in fact the origin of ``diffuse reflection''. Whether or not it
\textit{needs} to be explicitly modeled as subsurface scattering depends
on the scale of the scattering, including this scale relative to pixel
size. In some substances, the scattering is very local (presumably
because photons get absorbed too quickly to permit multiple scatterings
over longer ranges) and so the re-emergence of scattered photons from
the surface can be treated as Lambertian diffuse reflection. In other
(more translucent) substances, multiple scatterings without absorption
allow photons to re-emerge at large distances from the original
intersection point -- possibly with an asymmetric distribution, or even
on the other side of a sufficiently narrow object (thin edges of ears,
etc.).



\subsection{Reflection}

Reflection involves light scattering away from a surface. The \textit{amount} of light
that is scattered must always be $\le$ the incoming light; this can be
described using a single scalar $K_{r} \le 1$, or a spectrum of reflectance values.
For the standard RGB case, $\mathbf{K_{r}} = (K_{r,R}, K_{r,G}, K_{r,B})$.

If we imagine white light reflected diffusely from a surface, then
$\mathbf{L_{o}} = \mathbf{L_{i}} \mathbf{K_{r}} = (K_{r,R} L_{i,R},
K_{r,G} L_{i,G}, K_{r,B} L_{i,B})$, and $\mathbf{K_{r}}$ is then the
same as what we perceive as ``the'' color of the surface.


\section{Environment Maps}

\subsection{Cubic Maps}

The basic idea is that when a ray (reflected or otherwise) goes to infinity without
intersecting an object, you can give it a color $L$ from:
\begin{enumerate}
\item A constant background color value $L_{\rm back}$ [what we do currently]

\item A background/environment image.
\end{enumerate}

The simplest and most popular modern approach is to use a cubic map, where each of the
$+$ and $-$ axes has one face of the cube associated with it.

How do we choose which face of the map to use? Simple: identify the largest component
of the ray in question and take its sign. Thus, for a ray like $(x, y, z) = (0.5, -0.1, 0.9)$,
you would choose the $+z$ axis and its face; for $(x, y, z) = (-0.5, -0.1, -0.4)$, you
would choose the $-x$ axis; and so forth.

How do we choose where in the individual-face image to get a pixel value? We take the
complementary component values of the ray and convert them into scaled coordinates on that face,
using the value of the other component to scale them.

Thus, in the case of $(x, y, z) = (0.5, -0.1, 0.9)$, where we're using the map associated
with the $+z$ face, we take $x/z = 0.5/0.9$ and $y/z = -0.1/0.9$ to get values on the $(-1,1)$
interval; then we map these into the $(0,1)$ interval by adding 1 to each and then dividing the
result by 2:
\begin{equation}
u = (x/z + 1)/2, \; \; v = (y/z + 1)/2 .
\end{equation}

Now the problem is one of mapping values on the $(0,1)$ interval into pixel values
in the image $u$ and $v$ axes.


\end{document}
